\documentclass{exam}
\usepackage{babel}
\usepackage[utf8]{inputenc}
\usepackage{amsmath, amsthm, amssymb}
\usepackage{ragged2e}
\usepackage{blindtext}
\usepackage{lmodern}
\usepackage{tcolorbox}
\usepackage{hyperref}

\pagestyle{empty}
\renewcommand{\theenumi}{\alph{enumi}} 
\DeclareMathOperator*{\argmin}{arg\,min}
%\newcommand{\solvedByEnki}{$*^{E}$} 
%\newcommand{\solvedByRicardo}{$*^{R}$}

\begin{document}

\begin{center}
    \textbf{\Large Exercises 1.9}    
\end{center}

\section*{Exercise 1.9.1}
A factory has $n$ suppliers that produce quantities $x_1 \dots x_n$ per day. The factory is connected with suppliers by a system of roads, which
can be at variable capacities $c_1 \dots c_n$, so that the factory is supplied
daily the amount $x = c_1 x_1 + \dots + c_n x_n$.
\begin{enumerate}
    \item Given that the factory production process starts when the supply reaches the critical daily level $b$, write a formula for the daily factory revenue.
    \item Formulate the problem as a learning problem.
\end{enumerate}   

\section*{Exercise 1.9.2}
A number of finantial institutions, each having a wealth $x_i$, deposit amounts of money in a fund, at some adjustable rates of 
deposit $w_i$, so the money in the fund is given by $x = x_1 w_1 + \dots + x_n w_n$. The fund is \\
set up to function as in the following: as long a sthe fund has less than a certain reserve fund $M$, the fund manager does not invest. Only the
money exceeding the reserve found $M$ is invested. Let $k = e ^ {r t}$, where $r$ and $t$ denote the investment rate of return and time of investment, respectively.
\begin{enumerate}
    \item Find the formula for the investment.
    \item Formulate the problem as a learning problem.
\end{enumerate}

\section*{Exercise 1.9.3}
\begin{enumerate}
    \item Given a continous function $f: [0,1] \rightarrow \mathbb{R}$, find a linear function $L(x) = ax + b$ with $L(0) = f(0)$ and such that 
    $\frac{1}{2} \int^{1}_{0}(L(x) - f(x))^2 dx$ is minimized.
    \item  Given a continous function $f: [0,1] \times [0,1] \rightarrow \mathbb{R}$, find a linear function $L(x,y) = ax + by + c$ with $L(0,0) = f(0,0)$ and 
    such that the error $\frac{1}{2} \int_{[0,1]^2}(L(x,y) - f(x,y))^2 dx$ is minimized
\end{enumerate}

\section*{Exercise 1.9.4}
For any compact $K \subset \mathbb{R}^n$ we associate the symetric matrix $\displaystyle \rho_{ij} = \int_{K} x_i x_j d x_1 \dots d x_n$ The invertiblity of \\
\\
the matrix $(\rho_{ij})$ depends both on the shape of $K$ and the dimension $n$.
\begin{enumerate}
    \item Show that if $n = 2$ then $\text{det}(\rho_{ij}) \neq 0$, for any compact $K \subset \mathbb{R}^2$.
    \item Asume $K = [0,1]^n$. Show that $\text{det}(\rho_{ij}) \neq 0 \text{, for any $n \geq 1$}$.
\end{enumerate}

\newpage

\begin{center}    
    \section*{SOLUTIONS}
\end{center}

\subsection*{1.9.1 (a)}
Let $\pmb{\text{c}} := (c_1 ,\ldots, c_n)$ and $\pmb{\text{p}} := (x_1 ,\ldots, x_n)$ the roads variable capacities and the produced quantities, respectively. Then $x = \pmb{\text{c}} \cdot \pmb{\text{p}}$. Suppose the cost of product per item is $k$, so if the 
production starts after the critical daily level $b$ is meet. This is, $x - b > 0 $. It is clear that the revenue $L_{r}$ will be given by the formula: \\
\begin{equation*}
    \hspace*{-7cm} L_{r}(\pmb{\text{p}};\pmb{\text{c}},b) = 
        \left\{
        \begin{aligned}
            & k ( \pmb{\text{c}} \cdot \pmb{\text{p}} - b), & \text{if } \pmb{\text{c}} \cdot \pmb{\text{p}} - b > 0\\
            & 0,  \text{ otherwise}.
        \end{aligned}
        \right.
\end{equation*}

\subsection*{1.9.1 (b)}
The learning problem can be stated like this: "Given a vector of road variable capacities $\pmb{\text{c}}$ and a daily critical level $b$, provided that the production 
of the $n$ factories is expressed by the vector $\pmb{\text{p}}$. The goal is find a vector $\pmb{\text{p}}^\star$ and a scalar $b^\star$ such that the ideal revenue
$r(\pmb{\text{p}}) = \pmb{\text{c}} \cdot \pmb{\text{p}} $ is close that provided by the data $L_{r}(\pmb{\text{p}};\pmb{\text{c}},b)$ (obtained in 1.9.1 (a))". In other words, the pair $(\pmb{\text{c}}^\star,b^\star)$ minimizes the distance (in the $\mathcal{L}_2$ sense)
between $r(\pmb{\text{p}})$ and $L_{r}(\pmb{\text{p}};\pmb{\text{c}},b)$. In symbols: \\
\begin{equation*}
    \hspace*{-7cm}(\pmb{\text{c}}^\star,b^\star) = \argmin_{\tiny\pmb{\text{c}} \in \mathbb{R}^n \text{, } p \in \mathbb{R}} \displaystyle \int_{\mathcal{K}} (r(\pmb{\text{p}}) - L_{r}(\pmb{\text{p}};\pmb{\text{c}},b))^2 d \pmb{\text{p}}.
\end{equation*} 

\subsection*{1.9.2 (a)}
If $\pmb{\text{w}} := (w_1 \ldots w_n)$ encodes the adjustable rates of deposit corresponding to each of the $n$ finantial \\
institutions and $\pmb{\text{x}} := (x_1 \ldots x_n)$ the wealth of each of the $n$ institutions, the money in the fund is expressed by $x = \pmb{\text{w}} \cdot \pmb{\text{x}}$. It's known that the fund is set to function if the revenue exceds a given capital 
$M$ and that the investment grows proportional to $e^{r t}$(profit per investment), then it is clear that the invesment is given by the formula: \\
\begin{equation*}
    \hspace*{-7cm} L_{I}(\pmb{\text{x}};\pmb{\text{w}},M) = 
        \left\{
        \begin{aligned}
            & e^{r t} ( \pmb{\text{w}} \cdot \pmb{\text{x}} - M), \text{ if }   \pmb{\text{w}} \cdot \pmb{\text{x}}  > M \\
            & 0, \text{ otherwise}.
        \end{aligned}
        \right.
\end{equation*}

\subsection*{1.9.2 (b)}
Let $I(\pmb{\text{x}}) = x = \pmb{\text{w}} \cdot \pmb{\text{x}}$ be the ideal investment, $L_{I}(\pmb{\text{x}};\pmb{\text{w}},M)$ given in 1.9.2 (b). Then the learning problem 
can be stated as follows: "Given the vector of wealth of the $n$ institutions $\pmb{\text{x}}$, the vector of adjustable rates of deposit $\pmb{\text{w}}$ and that inversions are placed 
in the fund if the capital exceeds the quantity $M$. It is needed to find a tuple $(\pmb{\text{w}}^\star,M)$ that minimizes the distance (in the $\mathcal{L}_2$ sense) between the functions
$I(\pmb{\text{x}})$ and $L_{I}(\pmb{\text{x}};\pmb{\text{w}},M)$" \\
\\
In mathematical terms this means that: \begin{equation*} 
    (\pmb{\text{w}}^\star,M^\star) = \argmin_{\tiny\pmb{\text{w}} \in \mathbb{R}^n \text{, } M \in \mathbb{R}} \displaystyle \int_{\mathcal{K}} (I(\pmb{\text{x}}) - L_{I}(\pmb{\text{x}};\pmb{\text{c}},M))^2 d \pmb{\text{x}}.
\end{equation*} 

\subsection*{1.9.3 (a)}
It follows from the constrain $f(0) = L(0)$ that $b = f(0)$. From now on the notation $L(x;a)$ will be used. That said, the task is now to find an $a^\star$ such that\\
$\displaystyle a^\star = \argmin_{ \tiny a \in \mathbb{R}} {\lVert f(x) - L(x;a) \rVert}^2_{\mathcal{L}_2([0,1])}$. Now, let $C(a) := {\lVert f(x) - L(x;a) \rVert}^2_{\mathcal{L}_2([0,1])}$, then $C(a)$ has the most explicit form: 
\begin{equation*}
    C(a) = \displaystyle \int_{[0,1]} (f(x) - ax -f(0))^2 d x.
\end{equation*}
The extremizing $a^\star$ can be found by computing critical points and applying the second derivative test to determine the nature of the critical point. Then one has: \\

$\frac{d}{da} C(a) = \displaystyle\int_{[0,1]} -2(f(x) - ax -f(0)) x d x = a \displaystyle \int_{[0,1]} 2x^2 d x +  \displaystyle\int_{[0,1]} 2(f(0) - f(x)) x d x$.  To find the critical \\
point $a^\star$ we solve the equation $\frac{d}{da} C(a) = 0$. Solving for $a$ one finds that such extrema is meet at: 
\begin{equation*}
    \hspace{2cm}a^\star = \displaystyle\frac{\displaystyle\int_{[0,1]} (f(x) - f(0)) x d x}{\displaystyle \int_{[0,1]} x^2 d x}. \text{ Now, lets compute the second derivative to determine the nature}
\end{equation*} of this critical point. A straight-forward calculation for finding the second derivative yields: \newline
\\
$\frac{d^2}{d x^2} C(a) = 2 \displaystyle \int_{[0,1]} x^2 d x$. Because the even function $2x^2$ satisfies $2x^2 > 0 \text{, } \forall x \in [0,1]$ and the interval of integration is not symetric, is clear that $\frac{d^2}{d x^2} C(a) = 2 \displaystyle \int_{[0,1]} x^2 d x > 0 \text{, } \forall a \in \mathbb{R}$. From this follows 
that $a^\star$ is a minimum. Then, $L(x) = \displaystyle\frac{\displaystyle\int_{[0,1]} (f(t) - f(0)) t d t}{\displaystyle \int_{[0,1]} t^2 d t}x + f(0)$.

\subsection*{1.9.3 (b)}
The condition $f(0,0) = L(0,0)$ implies $b = f(0,0)$.Lets be more explicit by writing $L(x,y;a,b)$. The goal is \\
to find $(a^\star, b^\star)$ satisfying $\displaystyle (a^\star, b^\star) = \argmin_{ \tiny a \in \mathbb{R}, b \in \mathbb{R}} {\lVert f(x,y) - L(x,y;a,b) \rVert}^2_{\mathcal{L}_2([0,1]^2)}$. If $C(a,b) = {\lVert f(x,y) - L(x,y;a,b) \rVert}^2_{\mathcal{L}_2([0,1]^2)}$, it's necessary to solve the linear system: 
\\
\begin{equation}
    \hspace*{-10cm}
        \left\{
        \begin{aligned}
            & \frac{\partial}{\partial a} C(a,b) = 0\\
            & \frac{\partial}{\partial b} C(a,b) = 0.
        \end{aligned}
        \right.
\end{equation}
Lets now compute both partial derivatives. For $a$ we have: \\
\begin{equation*}
        \begin{aligned}
             \frac{\partial}{\partial a} C(a,b) &= \displaystyle\int_{[0,1]^2} -2(f(x,y) - ax -by-f(0,0)) x d x d y  \\
            &= a \displaystyle\int_{[0,1]^2} 2x^2 d x d y + b\displaystyle\int_{[0,1]^2} 2xy d x d y + \displaystyle\int_{[0,1]^2} 2(f(0,0) - f(x,y)) x d x d y 
        \end{aligned}
\end{equation*}
Likewise, for $b$ we find: 
\begin{equation*}
    \begin{aligned}
         \frac{\partial}{\partial b} C(a,b) &= \displaystyle\int_{[0,1]^2} -2(f(x,y) - ax -by-f(0,0)) y d x d y  \\
        &= a \displaystyle\int_{[0,1]^2} 2xy d x d y + b\displaystyle\int_{[0,1]^2} 2y^2 d x d y + \displaystyle\int_{[0,1]^2} 2(f(0,0) - f(x,y)) y d x d y.
    \end{aligned}
\end{equation*}
\\
System $(1)$ is equivalent to: 
\begin{equation}
    \left\{
    \begin{aligned}
        a \displaystyle\int_{[0,1]^2} 2x^2 d x d y + b \displaystyle\int_{[0,1]^2} 2xy d x d y &= \displaystyle\int_{[0,1]^2} 2(f(x,y) - f(0,0)) x d x d y  \\
        a \displaystyle\int_{[0,1]^2} 2xy d x d y + b \displaystyle\int_{[0,1]^2} 2y^2 d x d y &= \displaystyle\int_{[0,1]^2} 2(f(x,y) - f(0,0)) y d x d y \\
    \end{aligned}
    \right.
\end{equation}
Because the function $L(x,y;a,b)$ is continous, then the mixed partial derivatives coincide. It's easy to verify $\frac{\partial^2}{\partial a \partial b} C(a,b) =  \displaystyle\int_{[0,1]^2} 2xy d x d y$. Similarly, is 
straight-forward to see that \\
$\frac{\partial^2}{\partial a^2} C(a,b) = \displaystyle\int_{[0,1]^2} 2x^2 d x d y$  and $\frac{\partial^2}{\partial b^2} C(a,b) =  \displaystyle\int_{[0,1]^2} 2y^2 d x d y$. Then, system $(2)$ can be represented in the 
following matricial form:
\begin{equation}
        \displaystyle\begin{aligned}
        \begin{bmatrix}
            \displaystyle\int_{[0,1]^2} 2x^2 d x d y & \displaystyle\int_{[0,1]^2} 2xy d x d y\\
            \\
            \displaystyle\int_{[0,1]^2} 2xy d x d y & \displaystyle\int_{[0,1]^2} 2y^2 d x d y
        \end{bmatrix} 
        \begin{bmatrix}
            a\\
            \\
            b
        \end{bmatrix} 
        =
        \begin{bmatrix}
            \displaystyle\int_{[0,1]^2} 2(f(x,y) - f(0,0)) x d x d y\\
            \\
            \displaystyle\int_{[0,1]^2} 2(f(x,y) - f(0,0)) y d x d y
        \end{bmatrix} 
    \end{aligned}
\end{equation} 
Lets denote $H_{C}$ the matrix on the l.h.s of the equation $(3)$. By the \href{https://en.wikipedia.org/wiki/Cauchy%E2%80%93Schwarz_inequality}{Cauchy–Bunyakovsky–Schwarz inequality} 
follows that:\\
\\
$\text{det}( H_{C}) = 4 \displaystyle\int_{[0,1]^2} x^2 d x d y \displaystyle\int_{[0,1]^2} y^2 d x d y  - 4 (\displaystyle\int_{[0,1]^2} xy d x d y )^2 > 0$. Since $\text{det}( H_{C}) \neq 0$ there's a unique solution to the system $(3)$. Such solutions are given explicitly by:\\
\begin{equation}
    \begin{aligned}
        a^\star&=
        \displaystyle \frac{ 
            \text{det}\begin{bmatrix}
                \displaystyle\int_{[0,1]^2} x^2 d x d y & \displaystyle\int_{[0,1]^2} (f(x,y) - f(0,0)) x d x d y\\
                \\
                \displaystyle\int_{[0,1]^2} xy d x d y & \displaystyle\int_{[0,1]^2} (f(x,y) - f(0,0)) y d x d y
            \end{bmatrix}
            }
            {
              \displaystyle\int_{[0,1]^2} x^2 d x d y \displaystyle\int_{[0,1]^2} y^2 d x d y  - (\displaystyle\int_{[0,1]^2} xy d x d y )^2
            }\\
        \\
        b^\star &=
       \displaystyle \frac{ 
           \text{det}\begin{bmatrix}
                \displaystyle\int_{[0,1]^2} (f(x,y) - f(0,0)) x d x d y & \displaystyle\int_{[0,1]^2} xy d x d y \\
               \\
                \displaystyle\int_{[0,1]^2} (f(x,y) - f(0,0)) y d x d y & \displaystyle\int_{[0,1]^2} y^2 d x d y &
           \end{bmatrix}
           }
           {
             \displaystyle\int_{[0,1]^2} x^2 d x d y \displaystyle\int_{[0,1]^2} y^2 d x d y  - (\displaystyle\int_{[0,1]^2} xy d x d y )^2
           }
    \end{aligned}
\end{equation}\\
Note that $(1,1)$ minor of $\text{det}( H_{C})$ also has positive determinant, then by the \href{https://en.wikipedia.org/wiki/Sylvester%27s_criterion}{Sylvester's criterion} $H_{C}$ is positively defined, and the critical point $(a^\star,b^\star)$ is a minimum. 
\\
\subsection*{1.9.4 (a)}
For the $n = 2$ case we have:\\
\begin{equation*}
    \pmb{\text{P}} = \begin{bmatrix}
        \displaystyle\int_{K} x_1^2 d x_1 d x_2 & \displaystyle\int_{K} x_1 x_2 d x_1 d x_2\\
       \\
       \displaystyle\int_{K} x_1 x_2 d x_1 d x_2 & \displaystyle\int_{K} x_2^2 d x_1 d x_2 
    \end{bmatrix} \text{. Once again, by the Cauchy–Bunyakovsky–Schwarz inequality}
\end{equation*}
follows that $\text{det}(\pmb{\text{P}}) = \displaystyle\int_{K} x_1^2d x_1 d x_2  \displaystyle\int_{K} x_2^2 d x_1 d x_2  - (\displaystyle\int_{K} x_1 x_2 d x_1 d x_2 )^2 > 0$. The equality can not occur because that would imply $K$ is section of line which contradicts the arbitrairiness of $K$. \ $\qedsymbol{}$
\subsection*{1.9.4 (b)}
\end{document}