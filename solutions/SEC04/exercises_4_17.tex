\documentclass{exam}
\usepackage{babel}
\usepackage[utf8]{inputenc}
\usepackage{amsmath, amsthm, amssymb}
\usepackage{ragged2e}
\usepackage{lmodern}
\usepackage{tcolorbox}
\usepackage{hyperref}

\pagestyle{empty}
\renewcommand{\theenumi}{\alph{enumi}}

\DeclareMathOperator*{\argmin}{arg\,min}

\begin{document}

\begin{center}
    \textbf{\Large Exercises 4.17 }
\end{center}

\section*{Exercise 4.17.1}
Let $f(x_1,x_2) = e^{x_1}\sin(x_2)$, with $(x_1,x_2) \in (0,1) \times (0,\frac{\pi}{2})$.
\begin{enumerate}
    \item Show that $f$ is a harmonic function;
    \item Find $\lVert \nabla f \lVert$;
    \item Show that the equation $\nabla f = 0$ does not have any solutions;
    \item Find the maxima and minima for the function $f$.
\end{enumerate}

\section*{Exercise 4.17.2}
Consider the quadratic function $Q(\bold{x}) = \frac{1}{2} \bold{x}^{T} A x - b \bold{x}$, with A nonsingular square matrix of order $n$.
\begin{enumerate}
    \item Find the gradient $\lVert \nabla Q \lVert$;
    \item Write the gradient descent iteration;
    \item Find the Hessian $H_{Q}$;
    \item Write the iteration by Newton's formula and compute its limit.
\end{enumerate}

\section*{Exercise 4.17.3}
Let A be a nonsingular square matrix of order $n$ and $b \in \mathbb{R}^{n}$ a given vector. Consider the linear system $A\bold{x} = b$. The solution can be approximated using 
the following steps:\\
\begin{enumerate}
    \item 
    \item 
    \item 
\end{enumerate}

% % Solutions to other exercises in Chapter X...

\newpage

\begin{center}    
    \section*{SOLUTIONS}
\end{center}

\subsection*{Exercise x.y.1 (a)}
At vero eos et accusamus et iusto odio dignissimos ducimus qui blanditiis praesentium voluptatum deleniti atque corrupti quos 
dolores et quas molestias excepturi sint occaecati cupiditate non provident, similique sunt in culpa qui officia deserunt mollitia animi, 
id est laborum et dolorum fuga.
\begin{proof}
    \begin{equation*}     
        \begin{aligned}
            \sigma^{\prime\prime\prime}(x) &= \frac{d}{ d x} \frac{e^{x}-e^{2x}}{(1 + e^{x})^3} = \frac{(e^x - 2e^{2x})(1 + e^x)^3 - 3(1 + e^x)^2 e^{x}(e^x - e^{2x})}{(1 + e^x)^6}\\
            &= \frac{e^{x} \{ 1 - 4e^x + e^{2x} \}(1 + e^x)^2}{(1 + e^x)^{6}} = \frac{e^{x} \{ 1 - 4e^x + e^{2x} \}}{(1 + e^x)^{4}}
        \end{aligned}
\end{equation*}
\end{proof}
\subsection*{Exercise x.y.2 (a)}
Et harum quidem rerum facilis est et expedita distinctio. Nam libero tempore, cum soluta nobis est 
eligendi optio cumque nihil impedit quo minus id quod maxime placeat facere possimus, omnis voluptas assumenda est, 
omnis dolor repellendus. 
\begin{proof}
     a = a
\end{proof}
\end{document}