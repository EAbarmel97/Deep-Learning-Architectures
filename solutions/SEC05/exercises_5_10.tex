\documentclass{exam}
\usepackage{babel}
\usepackage[utf8]{inputenc}
\usepackage{amsmath, amsthm, amssymb}
\usepackage{ragged2e}
\usepackage{lmodern}
\usepackage{tcolorbox}
\usepackage{hyperref}
\usepackage{verbatim}

\pagestyle{empty}
\renewcommand{\theenumi}{\alph{enumi}}
\renewenvironment{proof}{{\noindent\itshape\ignorespaces}}{{\hfill$\qed$\\}}

\DeclareMathOperator*{\argmin}{arg\,min}

\begin{document}

\begin{center}
    \textbf{\Large Exercises 5.10 }
\end{center}

\section*{Exercise 5.10.1}
Recall that $\neg x$ is the negative of the Boolean variable $x$. 
\begin{enumerate}
    \item Show that a single perceptron can learn the Boolean function $y = x_{1}\land \neg x_{2}$, with some $x_{1} \text{ , } x_{2} \in \{0,1\}$.
    \item The same question as in part $a$ for the Boolean function $y = x_{1}\lor \neg x_{2}$, with some $x_{1} \text{ , } x_{2} \in \{0,1\}$.
    \item Show that a perceptron with one Boolean input, $x$, can learn the negation function $y = \neg x$. What about the linear neuron?
    \item Show that a perceptron with three Boolean inputs, $x_{1} \text{ , } x_{2} \text{ , } x_3$, can learn the negation function $y = \neg x$. What about $x_1 \lor x_2 \lor x_3$?
\end{enumerate}

\section*{Exercise 5.10.2}
Show that two finite linearly separable sets $A$ and $B$ can be separated by a perceptron with rational weights. 

% % Solutions to other exercises in Chapter X...

\newpage

\begin{center}    
    \section*{SOLUTIONS}
\end{center}

\begin{comment}
    \subsection*{Exercise x.y.1 (a)}
    At vero eos et accusamus et iusto odio dignissimos ducimus qui blanditiis praesentium voluptatum deleniti atque corrupti quos 
    dolores et quas molestias excepturi sint occaecati cupiditate non provident, similique sunt in culpa qui officia deserunt mollitia animi, 
    id est laborum et dolorum fuga.
    \begin{proof}
        \begin{equation*}     
            \begin{aligned}
                \sigma^{\prime\prime\prime}(x) &= \frac{d}{ d x} \frac{e^{x}-e^{2x}}{(1 + e^{x})^3} = \frac{(e^x - 2e^{2x})(1 + e^x)^3 - 3(1 + e^x)^2 e^{x}(e^x - e^{2x})}{(1 + e^x)^6}\\
                &= \frac{e^{x} \{ 1 - 4e^x + e^{2x} \}(1 + e^x)^2}{(1 + e^x)^{6}} = \frac{e^{x} \{ 1 - 4e^x + e^{2x} \}}{(1 + e^x)^{4}}
            \end{aligned}
    \end{equation*}
    \end{proof}
    \subsection*{Exercise x.y.2 (a)}
    Et harum quidem rerum facilis est et expedita distinctio. Nam libero tempore, cum soluta nobis est 
    eligendi optio cumque nihil impedit quo minus id quod maxime placeat facere possimus, omnis voluptas assumenda est, 
    omnis dolor repellendus. 
    \begin{proof}
         a = a
    \end{proof}
\end{comment}
\end{document}